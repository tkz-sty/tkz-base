\section{Style Use}

\subsection{Modification of \tkzname{\tkznameofpack}}

\tkzname{tkz-base.sty} has a default configuration file. Its existence is not
mandatory, but if it exists, you can modify it to get different default styles.
I only give a quick description of this file, as it may evolve soon.

In \tkzname{tkz-base.cfg}, you can set the axes, the reference (if used), the
grid, etc. as well as the styles which are linked to these objects.
It is possible to modify the styles of the points and segments.

It is also possible to define the dimensions of a drawing by default by
modifying \tkzname{xmin}, \tkzname{xmax}, \tkzname{ymin} and \tkzname{ymax}.

\begin{tkzltxexample}[small]
\def\tkz@xa{0}
\def\tkz@xb{10}
\def\tkz@ya{0}
\def\tkz@yb{10}
\end{tkzltxexample}

These lines are used to define the values of \tkzname{xmin}, \tkzname{xmax},
etc.

You can change them, for example:

\begin{tkzltxexample}[small]
\def\tkz@xa{-5}
\def\tkz@xb{-5}
\def\tkz@ya{5}
\def\tkz@yb{5}
\end{tkzltxexample}

Here's a list of used styles you'll find in \tkzname{tkz-base.cfg}

\begin{itemize}
\item xlabel style
\item xaxe style
\item ylabel style
\item yaxe style
\item rep style
\item line style
\item point style
\item mark style
\item compass style
\item vector style
\item arrow coord style
\item xcoord style
\item ycoord style
\end{itemize}

\subsection{Use \tkzcname{tikzset}}

It's better to use \tkzcname{tikzset} now rather than \tkzcname{tikzstyle}\ and
it's possible to use \tkzname{tkz-base.cfg}.

If you want to change the appearance of the axes of the  orthogonal coordinate
system, for example place arrows at each end or remove them. This can be done in
\tkzname{tkz-base.cfg} or in your code.

\begin{tkzltxexample}[small]
\tikzset{xaxe style/.style ={>=latex,<->}}
\end{tkzltxexample}

The transformation will be valid for the entire document. Note that
\tkzname{xmin} has been modified, in fact the arrow and the line corresponding
to the graduation merge.

\begin{tkzexample}[latex=7cm,small]
\tikzset{xaxe style/.style = {<->}}
\tikzset{xlabel style/.style={below=6pt}}
\begin{tikzpicture}
  \tkzInit[xmin=-0.5,xmax=5]
  \tkzDrawX
  \tkzLabelX
\end{tikzpicture}
\end{tkzexample}

\endinput
