\section{Texts and Legends}

\subsection{Placing a title}
Of course you can use \TIKZ, but the macro I propose to allow you to place the text using the units chosen for the drawing.

The options are always those of \TIKZ, in particular the following ones:
\begin{NewMacroBox}{tkzText}{\oarg{local options}\parg{dot}\var{text}}%
The point can either be given by its coordinates or by its name.

\begin{tabular}{lll}%
\toprule
options  & default & definition\\
\midrule
\TOline{color   }{|black|}{current colour}
\TOline{text    }{|black|}{text colour}
\TOline{fill    }{|white|}{background colour}
\TOline{opacity }{|1|    }{opacity}
\end{tabular}
\end{NewMacroBox}

\subsubsection{A title}

\begin{tkzexample}[latex=8cm]
\begin{tikzpicture}
  \tkzInit[xmax  = 6,   ymin  = 1000,%
           ymax  = 4000,ystep = 1000]
  \tkzAxeXY
  \tkzText[draw,
           line width = 1pt,%
           color    = red,%
           fill = orange!20](3,4000)%
           {Revenue by month}
\end{tikzpicture}
\end{tkzexample}

\subsubsection{Draft}

\begin{tkzexample}[latex=8cm]
\begin{tikzpicture}
  \tkzInit[xmax  = 6,   ymin  = 1000,%
           ymax  = 4000,ystep = 1000]
  \tkzGrid   \tkzAxeXY
  \tkzText[draw,opacity=.2,
           rotate=45,inner sep=.6 cm,
           line width = 1pt,
           color = black,
           fill = purple!20](3,2500)
           {\Huge DRAFT}
\end{tikzpicture}
\end{tkzexample}

\subsubsection{Text with a point}
It is possible to give the reference of a point instead of its coordinates.

\begin{tkzexample}[latex=8cm]
\begin{tikzpicture}
  \tkzInit[ymax=5,xmax=6]
  \tkzAxeXY
  \tkzDefPoint(3,3){A}
  \tkzText[draw,opacity=.6,
           inner sep=.6 cm,
           line width = 1pt,
           color    = black,
           fill = purple!20](A)
           {My text}
\end{tikzpicture}
\end{tkzexample}

\subsubsection{Text format}
 The option \tkzname{text width} is interesting, see the pgfmanual for more information.
\begin{tkzexample}[latex=8cm,small]
\begin{tikzpicture}
 \tkzInit[ymax=5,xmax=6]
 \tkzAxeXY
 \tkzText[draw,opacity=.6,
          inner sep=.6 cm,
          line width = 1pt,
          color    = black,
          fill = purple!20,
          text width=3cm](3,3)
          {My text\\\ Reference}
\end{tikzpicture}
\end{tkzexample}

\subsection{Placing legends}
There are two ways to use this macro. Either you can place legends for curves. Then you can represent lines with their own style, or you can differentiate symbols (mark).
\hypertarget{legend}{}
\begin{NewMacroBox}{tkzLegend}{\oarg{local options}\var{mark/color/size/text}}%
The arguments differ according to the boolean \tkzname{line}.

\medskip
\begin{tabular}{lll}%
\toprule
options  & default & definition\\
\midrule
\TOline{line}{false}{Boolean: line or symbol}
\end{tabular}

With |line=true|

\begin{tabular}{lll}
\toprule
arguments  & default & example\\
\midrule
\TAline{style/line width/color/text}{no default}{dashed/1pt/red/Product Recipe B}
\end{tabular}

\medskip
With |line=false|

\begin{tabular}{lll}
\toprule
arguments  & default & example\\
\midrule
\TAline{mark/mark size/color/text}{no default}{heart/1ex/red!30/Product Recipe A}
\end{tabular}
\end{NewMacroBox}

\subsubsection{Legends with symbols}
\begin{tkzexample}[vbox]
\begin{tikzpicture}
\tkzInit[xmax=12,ymin=1000,ymax=11000,ystep=2000]
\tkzGrid[color=orange!30]
\tkzDrawX[below right,label=Mois]
\tkzDrawY[above left,label=Recette]
\tkzLabelX
\tkzLabelY
\tkzDefSetOfPoints{1/2000,2/3000,3/2000,4/2500,5/4200,6/4800,7/4600,
                   8/5200,9/6200,10/7000,11/7400,12/10000}
\tkzDefSetOfPoints[prefix=P]{1/3200,2/4100,3/3300,4/3300,5/5000,6/5500,7/5200,8/4000,
         9/3000,10/6000,11/8400,12/9000}
\tkzSetUpMark[mark=heart,color=black,fill=red!30,size=4pt]
\tkzJoinSetOfPoints[thick,color=brown,double]
\tkzDrawSetOfPoints
\tkzJoinSetOfPoints[prefix=P,thick,color=orange,double]
\tkzDrawSetOfPoints[prefix=P,mark=square*,mark size=4pt,
                    mark options={color=blue,fill=blue!40}]
\tkzLegend[draw,rounded corners,fill=orange!20,text=brown,
          line width=2pt](5,10000){heart/1ex/red!30/Product Recipe A,%
                                   square*/0.75ex/blue!40/Product Recipe B}
\end{tikzpicture}
\end{tkzexample}



\endinput
